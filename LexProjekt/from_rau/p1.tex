Nicolaus Copernicus (German: Nikolaus Kopernikus; Italian: Nicolo Copernico; Polish: About this sound Miko�aj Kopernik (help�info)) (19 February 1473 � 24 May 1543) was a Renaissance astronomer and the first person to formulate a comprehensive heliocentric cosmology which displaced the Earth from the center of the universe.[1]

Copernicus' epochal book, De revolutionibus orbium coelestium (On the Revolutions of the Celestial Spheres), published just before his death in 1543, is often regarded as the starting point of modern astronomy and the defining epiphany that began the scientific revolution. His heliocentric model, with the Sun at the center of the universe, demonstrated that the observed motions of celestial objects can be explained without putting Earth at rest in the center of the universe. His work stimulated further scientific investigations, becoming a landmark in the history of science that is often referred to as the Copernican Revolution.

Among the great polymaths of the Renaissance, Copernicus was a mathematician, astronomer, jurist with a doctorate in law, physician, quadrilingual polyglot, classics scholar, translator, artist,[2] Catholic cleric, governor, diplomat and economist.


% Miko�aj Kopernik (�ac. Nicolaus Copernicus[1], niem. Nikolaus Kopernikus; ur. 19 lutego 1473 w Toruniu, zm. 24 maja 1543 we Fromborku) � polski astronom, autor dzie�a De revolutionibus orbium coelestium[2] przedstawiaj�cego szczeg�owo i w naukowo u�ytecznej formie heliocentryczn� wizj� Wszech�wiata. Wprawdzie koncepcja heliocentryzmu pojawi�a si� ju� w staro�ytnej Grecji (jej tw�rc� by� Arystarch z Samos[3]), to jednak dopiero dzie�o Kopernika dokona�o prze�omu i wywo�a�o jedn� z najwa�niejszych rewolucji naukowych od czas�w staro�ytnych, nazywan� przewrotem kopernika�skim[4].

% Od 1497 roku sprawowa� funkcj� kanonika warmi�skiego, od 1503 scholastyka wroc�awskiego, a od 1511 kanclerza kapitu�y warmi�skiej.

% By� wybitnym polihistorem Renesansu, zajmowa� si� mi�dzy innymi astronomi�, matematyk�, prawem, ekonomi�, strategi� wojskow�, astrologi�[5][6], by� tak�e lekarzem oraz t�umaczem.
